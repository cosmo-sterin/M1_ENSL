


\date{\today}

\documentclass[12pt]{article}
\usepackage[utf8]{inputenc}
\usepackage[francais]{babel}
\usepackage{amsmath}
\usepackage{amssymb}

\title{Régression Linéaire et Généralisation}
\begin{document}
\maketitle


\section{Contexte}
\subsection{Régression Linéaire, Erreur de Généralisation}
Soit $f: E \subset [0,1]^d \to [0,1]$ une fonction inconnue avec E \textbf{fini}.   
On suppose que l'on connaît $f$ en quelques points :     
On dispose de $D = \{(x_i,y_i) \, | \, y_i = f(x_i)\}$ fini avec $m = |D| << |E|$.   
Le but de la \textbf{régression linéaire} est d'approcher $f$ au mieux avec $f_\theta$, modèle linéaire, défini par :   

$$ f_\theta(x) = \theta^{T}x$$
Avec $\theta \in \mathbb{R^d}$ le paramètre de $f_\theta$.    
Plus formellement on veut trouver $\theta$ qui minimise \textbf{l'erreur absolue} :   
$$ J(\theta) = \sum_{x \in E} (f(x)-f_\theta(x))^2 $$    
Cependant comme on ne connaît pas $E$, on doit se contenter d'approcher $J$ par $\hat J$ \textbf{l'erreur empirique} sur $D$ :
$$ \hat{J}(\theta) =  \sum_{i} (y_i-f_\theta(x_i))^2 = \sum_{i} (y_i-\theta^{T}x_i)^2 $$ 
1) Que signifie $J(\theta) = 0$ ? $\hat{J}(\theta) = 0$ ?   \\
2) Montrer $$ \hat{J}(\theta) \leq J(\theta)$$     
3) Minimiser $\hat{J}(\theta)$ revient-il à minimiser $J(\theta)$ ?   \\
Dans l'idéal on voudrait donc minimiser $J(\theta)$, cependant c'est $\hat{J}(\theta)$ que l'on peut minimiser en pratique.   
Ce qui nous intéresserait ce serait trouver un terme $M > 0$ tel que:    
$$ J(\theta^*) \leq \hat{J}(\theta^*) + M$$   
Avec $\theta^*$ qui minimise $\hat{J}$.  \\  
4) En quoi peut-on qualifier $M$ \textbf{d'erreur de généralisation} ?
\subsection{Régression Linéaire Ridge-Régularisée}
Pour des raisons passées sous silence, on considère ici les minimisations de : 
$$ J_r(\theta) = J(\theta) + \lambda ||\theta||_2^2 $$
$$\hat{J}_r(\theta) = \hat{J}(\theta) + \lambda ||\theta||_2^2$$   
Avec $\lambda$ un scalaire donné.   
On parle alors de \textbf{régression linéaire ridge-régularisée}.  
On admet qu'il existe un algorithme capable de fournir $\theta^{*}$ qui minimise $\hat{J}_r $.   
On cherche toujours une inégalité du type :   
$$ J_r(\theta^{*}) \leq \hat{J}_r(\theta^{*}) + M$$   
\section{ Stabilité Uniforme }   
Dans la suite on note:   
\begin{itemize}
\item $\theta$ qui minimise $\hat{J}_r$ sur $D$ 
\item $\theta^{ \backslash i}$  qui minimise $\hat{J}_r$ sur $D \backslash \{(x_i,y_i)\}$       
   
\end{itemize} \ \\

Pour trouver la borne $M$ on dispose du théorème suivant :  \\ 
\textbf{Théorème :} S'il existe $\beta$ tel que :  
$$ \forall i \in \{1 \dots m \},  \quad sup_{x \in E} |l(\theta,x)-l(\theta^{ \backslash i},x)| \leq \frac{\beta}{m}$$   
Avec $l(\theta,x) = (\theta^{T} x - f(x))^2$ et    $l(\theta^{\backslash i},x) = (\theta^\backslash{i}^{T} x - f(x))^2$    \\
Et si $l(\theta,x)$ \textbf{est bornée} par $M$  \\
\textbf{Alors :}    \\
Avec probabilité $1-\delta$:     
$$ J_r(\theta) \leq \hat{J}_r(\theta) + 2 \frac{\beta}{m} + (4\beta + M)\sqrt{\frac{ln(\frac{1}{\delta})}{2m}} $$   
5) Que se passe-t-il lorsque m est grand ? Comment l'interprétez vous ?

\section{Trouvons $\beta$}
Nous avons toujours    
\begin{itemize}
\item $\theta$ qui minimise $\hat{J}_r$ sur $D$ 
\item $\theta^{ \backslash i}$  qui minimise $\hat{J}_r$ sur $D \backslash \{(x_i,y_i)\}$       
   
\end{itemize} \ \\
6) Montrer que $||\theta||^2_2 \leq \frac{1}{\lambda}$    \\
7) En déduire $l(\theta,x) \leq 1$      \\ \\
    
On remplit donc la deuxième condition du théorème avec M=1.    
Passons à la première.   \\ \\
\noindent    
8) Montrer que $l$ est convexe en son premier argument   \\
9) Montrer qu'il existe $\sigma > 0$ tel que :
$$ \forall \theta_1,\theta_2 \, \forall x \quad |l(\theta_1,x)-l(\theta_2,x)| \leq \sigma |(\theta_1 - \theta_2)^T x|$$      
10) En déduire ce \textbf{lemme} :      
$$ \forall t \in [0,1] \quad ||\theta||^2_2 - ||\theta + t \Delta \theta||_2^2 + ||\theta^{ \backslash i}||_2^2 - ||\theta^{ \backslash i} - t \Delta \theta||_2^2 \leq \frac{t \sigma}{\lambda m}|\Delta \theta^T x_i| $$   
Avec $\Delta \theta = \theta^{ \backslash i} - \theta$.    
11) Constater que pour $t=\frac{1}{2}$ on a $||\theta||^2_2 - ||\theta + t \Delta \theta||_2^2 + ||\theta^{ \backslash i}||_2^2 - ||\theta^{ \backslash i} - t \Delta \theta||_2^2 = \frac{1}{2}||\Delta \theta||^2_2$   \\ \\
\noindent
12) En déduire une majoration de : 
$$ |l(\theta,x)-l(\theta^{ \backslash i},x)| $$    
13) Conclure

\section{Correction}
Coming soon

\section{Sources}
\begin{itemize}
\item Sujet (adapté) : $http://perso.univ-st-etienne.fr/habrarda/ENSML/td_uniform_stability.pdf$     
\item Correction : $http://perso.univ-st-etienne.fr/habrarda/ENSML/recap_uniformStability.pdf $       
\item Papier d'origine : 
Olivier Bousquet, André Elisseeff: Stability and Generalization. Journal of Machine Learning Research 2: 499-526 (2002)
\end{itemize}


\end{document}